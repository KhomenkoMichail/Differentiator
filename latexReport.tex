\documentclass{article}
\usepackage[utf8]{inputenc}
\usepackage[russian]{babel}
\usepackage{amsmath}
\usepackage{amssymb}
\usepackage{geometry}
\geometry{a4paper, margin=0.5cm}
\begin{document}
Все что мы имеем:
\[ f = 2 \cdot x + 3 \cdot x \]

Ввиду нехитрых преобразований:
\[ \frac{d}{dx}(x) = 1\]
Внимательный читатель заметит, что
\[ \frac{d}{dx}(3) = 0\]
Из леммы 6.66, следует, что:
\[ \frac{d}{dx}(3 \cdot x) = 0 \cdot x + 3 \cdot 1\]
Произведя некоторые подстановки:
\[ \frac{d}{dx}(x) = 1\]
Не умаляя общности:
\[ \frac{d}{dx}(2) = 0\]
Очевидно что:
\[ \frac{d}{dx}(2 \cdot x) = 0 \cdot x + 2 \cdot 1\]
Согласно школьной программе:
\[ \frac{d}{dx}(2 \cdot x + 3 \cdot x) = 0 \cdot x + 2 \cdot 1 + 0 \cdot x + 3 \cdot 1\]
Наведем косметики в функции:
\[ f = 0 \cdot x + 2 \cdot 1 + 0 \cdot x + 3 \cdot 1 \]

Произведя некоторые подстановки:
\[ 2 \cdot 1 = 2\]
Не умаляя общности:
\[ 3 \cdot 1 = 3\]
Не умаляя общности:
\[0 \cdot x = 0 \]

Внимательный читатель заметит, что
\[0 + 2 = 2 \]

Нетрудно заметить, что:
\[0 \cdot x = 0 \]

Не умаляя общности:
\[0 + 3 = 3 \]

Ввиду нехитрых преобразований:
\[ 2 + 3 = 5\]
Итого:
\[ 5 \]

\end{document}
