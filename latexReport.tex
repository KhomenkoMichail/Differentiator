\documentclass{article}
\usepackage[utf8]{inputenc}
\usepackage[russian]{babel}
\usepackage{amsmath}
\usepackage{amssymb}
\usepackage{geometry}
\geometry{a4paper, margin=0.5cm}
\begin{document}
Все что мы имеем:
\[ f = \sin(x) \cdot \cos(x) \]

Ввиду нехитрых преобразований:
\[ \frac{d}{dx}(x) = 1\]
Не умаляя общности:
\[ \frac{d}{dx}(\cos(x)) = -1 \cdot \sin(x) \cdot 1\]
Все доказано:
\[ \frac{d}{dx}(x) = 1\]
Внимательный читатель заметит, что
\[ \frac{d}{dx}(\sin(x)) = \cos(x) \cdot 1\]
Не умаляя общности:
\[ \frac{d}{dx}(\sin(x) \cdot \cos(x)) = \cos(x) \cdot 1 \cdot \cos(x) + \sin(x) \cdot -1 \cdot \sin(x) \cdot 1\]
\end{document}
