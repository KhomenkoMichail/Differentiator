\documentclass{article}
\usepackage[T2A]{fontenc}
\usepackage[utf8]{inputenc}
\usepackage[russian,english]{babel}
\usepackage{amsmath}
\usepackage{amssymb}
\usepackage{graphicx}
\usepackage{geometry}
\geometry{a4paper, margin=0.5cm}
\begin{document}

\begin{center}
\vspace*{3cm}

{\Huge \textbf{Курс \guillemotleftОт раздолб до отл 10 на семестрой по матану за одну ночь\guillemotright.}}\\[1cm]
{\Large Автор: Хоменко М.М.}\\[0.5cm]
\vfill
{\Large Долгопрудный, \the\year}
\end{center}
\newpage

\section{Введение}
Семестровая по матану лишь одно из жизненных испытаний, которое вам предстоит пройти. Будьте уверены: каждому, кто учился в школе, по силам сдать семестровую. Все задания составлены на основе школьной программы. Поэтому каждый из вас может успешно сдать семестровую.

Автор данного пособия не учился в школе и, дабы помочь себе подобным сдать экзамен, собрал небольшой курс упражнений для подготовки всего за одну ночь. Желаю Вам приятного времяпрепровождения.

\section{Упражнение первое: взятие производной простейшей функции}
Имеем функцию:
\[\boxed{ f = \sin(x) }\]

 Ее график имеет вид:

\begin{figure}[h]
	\centering
	\includegraphics[width=0.5\textwidth]{FUNC_GRAPHS/funcGraph1.png}
	\caption{график исходной функции}
\end{figure}
{\Large Вычислим производную данной функции:}

Очевидно что:
\[ \frac{d}{dx}(x) = 1\]
Согласно школьной программе:
\[ \frac{d}{dx}(\sin(x)) = \cos(x) \cdot 1\]
{\Large Теперь упростим полученную производную:}

Наведем косметики в функции:
\[ f = \cos(x) \cdot 1 \]

Произведя некоторые подстановки:
\[\cos(x) \cdot 1 = \cos(x) \]

Итого:
\[ \cos(x) \]

{\Large Итого получаем:}

\[\boxed{ \frac{d}{dx}(\sin(x)) = \cos(x)}\]
{\Large График полученной производной:}

\begin{figure}[h]
	\centering
	\includegraphics[width=0.5\textwidth]{FUNC_GRAPHS/funcGraph2.png}
	\caption{график производной}
\end{figure}
\section{Упражнение второе: вычисление касательной функции в точке}
Уравнение касательной функции:
\[ f = \sin(x) \]

 в точке x = 2:
\[\boxed{ f = 0.909297 + (-0.416147)*(x - 2)}\]

\begin{figure}[h]
	\centering
	\includegraphics[width=0.5\textwidth]{FUNC_GRAPHS/funcGraph3.png}
	\caption{График касательной функции в точке}
\end{figure}
\section{Упражнение третье - разложение по Тайлеру}{\Large Ну что почесались:}

Согласно школьной программе:
\[ \frac{d}{dx}(x) = 1\]
Все доказано:
\[ \frac{d}{dx}(\sin(x)) = \cos(x) \cdot 1\]
Все доказано:
\[ \frac{d}{dx}(1) = 0\]
Для любых из 6 СПС, верно, что:
(Желаю всем, кто пишет "СПС" вместо "спасибо"
продуктивно потратить сэкономленное время)
\[ \frac{d}{dx}(x) = 1\]
Очевидно что:
\[ \frac{d}{dx}(\cos(x)) = (-1) \cdot \sin(x) \cdot 1\]
Нетрудно заметить, что:
\[ \frac{d}{dx}(\cos(x) \cdot 1) = (-1) \cdot \sin(x) \cdot 1 \cdot 1 + \cos(x) \cdot 0\]
Очевидно что:
\[ \frac{d}{dx}(0) = 0\]
Ввиду нехитрых преобразований:
\[ \frac{d}{dx}(x) = 1\]
Не умаляя общности:
\[ \frac{d}{dx}(\cos(x)) = (-1) \cdot \sin(x) \cdot 1\]
Внимательный читатель заметит, что
\[ \frac{d}{dx}(\cos(x) \cdot 0) = (-1) \cdot \sin(x) \cdot 1 \cdot 0 + \cos(x) \cdot 0\]
Внимательный читатель заметит, что
\[ \frac{d}{dx}(1) = 0\]
Очевидно что:
\[ \frac{d}{dx}(1) = 0\]
Из леммы 6.66, следует, что:
\[ \frac{d}{dx}(x) = 1\]
Из леммы 6.66, следует, что:
\[ \frac{d}{dx}(\sin(x)) = \cos(x) \cdot 1\]
Для любых из 6 СПС, верно, что:
(Желаю всем, кто пишет "СПС" вместо "спасибо"
продуктивно потратить сэкономленное время)
\[ \frac{d}{dx}((-1)) = 0\]
Для любых из 6 СПС, верно, что:
(Желаю всем, кто пишет "СПС" вместо "спасибо"
продуктивно потратить сэкономленное время)
\[ \frac{d}{dx}((-1) \cdot \sin(x)) = 0 \cdot \sin(x) + (-1) \cdot \cos(x) \cdot 1\]
Из леммы 6.66, следует, что:
\[ \frac{d}{dx}((-1) \cdot \sin(x) \cdot 1) = (0 \cdot \sin(x) + (-1) \cdot \cos(x) \cdot 1) \cdot 1 + (-1) \cdot \sin(x) \cdot 0\]
Для любых из 6 СПС, верно, что:
(Желаю всем, кто пишет "СПС" вместо "спасибо"
продуктивно потратить сэкономленное время)
\[ \frac{d}{dx}((-1) \cdot \sin(x) \cdot 1 \cdot 1) = ((0 \cdot \sin(x) + (-1) \cdot \cos(x) \cdot 1) \cdot 1 + (-1) \cdot \sin(x) \cdot 0) \cdot 1 + (-1) \cdot \sin(x) \cdot 1 \cdot 0\]
Для любых из 6 СПС, верно, что:
(Желаю всем, кто пишет "СПС" вместо "спасибо"
продуктивно потратить сэкономленное время)
\[ \frac{d}{dx}((-1) \cdot \sin(x) \cdot 1 \cdot 1 + \cos(x) \cdot 0) = ((0 \cdot \sin(x) + (-1) \cdot \cos(x) \cdot 1) \cdot 1 + (-1) \cdot \sin(x) \cdot 0) \cdot 1 + (-1) \cdot \sin(x) \cdot 1 \cdot 0 + (-1) \cdot \sin(x) \cdot 1 \cdot 0 + \cos(x) \cdot 0\]
Для любых из 6 СПС, верно, что:
(Желаю всем, кто пишет "СПС" вместо "спасибо"
продуктивно потратить сэкономленное время)
\[ \frac{d}{dx}(0) = 0\]
Нетрудно заметить, что:
\[ \frac{d}{dx}(x) = 1\]
Произведя некоторые подстановки:
\[ \frac{d}{dx}(\cos(x)) = (-1) \cdot \sin(x) \cdot 1\]
Нетрудно заметить, что:
\[ \frac{d}{dx}(\cos(x) \cdot 0) = (-1) \cdot \sin(x) \cdot 1 \cdot 0 + \cos(x) \cdot 0\]
Внимательный читатель заметит, что
\[ \frac{d}{dx}(0) = 0\]
Произведя некоторые подстановки:
\[ \frac{d}{dx}(1) = 0\]
Произведя некоторые подстановки:
\[ \frac{d}{dx}(x) = 1\]
Произведя некоторые подстановки:
\[ \frac{d}{dx}(\sin(x)) = \cos(x) \cdot 1\]
Нетрудно заметить, что:
\[ \frac{d}{dx}((-1)) = 0\]
Очевидно что:
\[ \frac{d}{dx}((-1) \cdot \sin(x)) = 0 \cdot \sin(x) + (-1) \cdot \cos(x) \cdot 1\]
Очевидно что:
\[ \frac{d}{dx}((-1) \cdot \sin(x) \cdot 1) = (0 \cdot \sin(x) + (-1) \cdot \cos(x) \cdot 1) \cdot 1 + (-1) \cdot \sin(x) \cdot 0\]
Нетрудно заметить, что:
\[ \frac{d}{dx}((-1) \cdot \sin(x) \cdot 1 \cdot 0) = ((0 \cdot \sin(x) + (-1) \cdot \cos(x) \cdot 1) \cdot 1 + (-1) \cdot \sin(x) \cdot 0) \cdot 0 + (-1) \cdot \sin(x) \cdot 1 \cdot 0\]
Нетрудно заметить, что:
\[ \frac{d}{dx}((-1) \cdot \sin(x) \cdot 1 \cdot 0 + \cos(x) \cdot 0) = ((0 \cdot \sin(x) + (-1) \cdot \cos(x) \cdot 1) \cdot 1 + (-1) \cdot \sin(x) \cdot 0) \cdot 0 + (-1) \cdot \sin(x) \cdot 1 \cdot 0 + (-1) \cdot \sin(x) \cdot 1 \cdot 0 + \cos(x) \cdot 0\]
Нетрудно заметить, что:
\[ \frac{d}{dx}(0) = 0\]
Из леммы 6.66, следует, что:
\[ \frac{d}{dx}(1) = 0\]
Внимательный читатель заметит, что
\[ \frac{d}{dx}(x) = 1\]
Все доказано:
\[ \frac{d}{dx}(\sin(x)) = \cos(x) \cdot 1\]
Из леммы 6.66, следует, что:
\[ \frac{d}{dx}((-1)) = 0\]
Не умаляя общности:
\[ \frac{d}{dx}((-1) \cdot \sin(x)) = 0 \cdot \sin(x) + (-1) \cdot \cos(x) \cdot 1\]
Внимательный читатель заметит, что
\[ \frac{d}{dx}((-1) \cdot \sin(x) \cdot 1) = (0 \cdot \sin(x) + (-1) \cdot \cos(x) \cdot 1) \cdot 1 + (-1) \cdot \sin(x) \cdot 0\]
Нетрудно заметить, что:
\[ \frac{d}{dx}((-1) \cdot \sin(x) \cdot 1 \cdot 0) = ((0 \cdot \sin(x) + (-1) \cdot \cos(x) \cdot 1) \cdot 1 + (-1) \cdot \sin(x) \cdot 0) \cdot 0 + (-1) \cdot \sin(x) \cdot 1 \cdot 0\]
Для любых из 6 СПС, верно, что:
(Желаю всем, кто пишет "СПС" вместо "спасибо"
продуктивно потратить сэкономленное время)
\[ \frac{d}{dx}(1) = 0\]
Не умаляя общности:
\[ \frac{d}{dx}(0) = 0\]
Не умаляя общности:
\[ \frac{d}{dx}(x) = 1\]
Для любых из 6 СПС, верно, что:
(Желаю всем, кто пишет "СПС" вместо "спасибо"
продуктивно потратить сэкономленное время)
\[ \frac{d}{dx}(\sin(x)) = \cos(x) \cdot 1\]
Из леммы 6.66, следует, что:
\[ \frac{d}{dx}((-1)) = 0\]
Все доказано:
\[ \frac{d}{dx}((-1) \cdot \sin(x)) = 0 \cdot \sin(x) + (-1) \cdot \cos(x) \cdot 1\]
Нетрудно заметить, что:
\[ \frac{d}{dx}((-1) \cdot \sin(x) \cdot 0) = (0 \cdot \sin(x) + (-1) \cdot \cos(x) \cdot 1) \cdot 0 + (-1) \cdot \sin(x) \cdot 0\]
Для любых из 6 СПС, верно, что:
(Желаю всем, кто пишет "СПС" вместо "спасибо"
продуктивно потратить сэкономленное время)
\[ \frac{d}{dx}(1) = 0\]
Произведя некоторые подстановки:
\[ \frac{d}{dx}(1) = 0\]
Не умаляя общности:
\[ \frac{d}{dx}(x) = 1\]
Очевидно что:
\[ \frac{d}{dx}(\cos(x)) = (-1) \cdot \sin(x) \cdot 1\]
Согласно школьной программе:
\[ \frac{d}{dx}(\cos(x) \cdot 1) = (-1) \cdot \sin(x) \cdot 1 \cdot 1 + \cos(x) \cdot 0\]
Все доказано:
\[ \frac{d}{dx}((-1)) = 0\]
Для любых из 6 СПС, верно, что:
(Желаю всем, кто пишет "СПС" вместо "спасибо"
продуктивно потратить сэкономленное время)
\[ \frac{d}{dx}((-1) \cdot \cos(x) \cdot 1) = 0 \cdot \cos(x) \cdot 1 + (-1) \cdot ((-1) \cdot \sin(x) \cdot 1 \cdot 1 + \cos(x) \cdot 0)\]
Для любых из 6 СПС, верно, что:
(Желаю всем, кто пишет "СПС" вместо "спасибо"
продуктивно потратить сэкономленное время)
\[ \frac{d}{dx}(x) = 1\]
Из леммы 6.66, следует, что:
\[ \frac{d}{dx}(\sin(x)) = \cos(x) \cdot 1\]
Ввиду нехитрых преобразований:
\[ \frac{d}{dx}(0) = 0\]
Согласно школьной программе:
\[ \frac{d}{dx}(0 \cdot \sin(x)) = 0 \cdot \sin(x) + 0 \cdot \cos(x) \cdot 1\]
Не умаляя общности:
\[ \frac{d}{dx}((0 \cdot \sin(x) + (-1) \cdot \cos(x) \cdot 1)) = 0 \cdot \sin(x) + 0 \cdot \cos(x) \cdot 1 + 0 \cdot \cos(x) \cdot 1 + (-1) \cdot ((-1) \cdot \sin(x) \cdot 1 \cdot 1 + \cos(x) \cdot 0)\]
Очевидно что:
\[ \frac{d}{dx}((0 \cdot \sin(x) + (-1) \cdot \cos(x) \cdot 1) \cdot 1) = (0 \cdot \sin(x) + 0 \cdot \cos(x) \cdot 1 + 0 \cdot \cos(x) \cdot 1 + (-1) \cdot ((-1) \cdot \sin(x) \cdot 1 \cdot 1 + \cos(x) \cdot 0)) \cdot 1 + (0 \cdot \sin(x) + (-1) \cdot \cos(x) \cdot 1) \cdot 0\]
Согласно школьной программе:
\[ \frac{d}{dx}(((0 \cdot \sin(x) + (-1) \cdot \cos(x) \cdot 1) \cdot 1 + (-1) \cdot \sin(x) \cdot 0)) = (0 \cdot \sin(x) + 0 \cdot \cos(x) \cdot 1 + 0 \cdot \cos(x) \cdot 1 + (-1) \cdot ((-1) \cdot \sin(x) \cdot 1 \cdot 1 + \cos(x) \cdot 0)) \cdot 1 + (0 \cdot \sin(x) + (-1) \cdot \cos(x) \cdot 1) \cdot 0 + (0 \cdot \sin(x) + (-1) \cdot \cos(x) \cdot 1) \cdot 0 + (-1) \cdot \sin(x) \cdot 0\]
Очевидно что:
\[ \frac{d}{dx}(((0 \cdot \sin(x) + (-1) \cdot \cos(x) \cdot 1) \cdot 1 + (-1) \cdot \sin(x) \cdot 0) \cdot 1) = ((0 \cdot \sin(x) + 0 \cdot \cos(x) \cdot 1 + 0 \cdot \cos(x) \cdot 1 + (-1) \cdot ((-1) \cdot \sin(x) \cdot 1 \cdot 1 + \cos(x) \cdot 0)) \cdot 1 + (0 \cdot \sin(x) + (-1) \cdot \cos(x) \cdot 1) \cdot 0 + (0 \cdot \sin(x) + (-1) \cdot \cos(x) \cdot 1) \cdot 0 + (-1) \cdot \sin(x) \cdot 0) \cdot 1 + ((0 \cdot \sin(x) + (-1) \cdot \cos(x) \cdot 1) \cdot 1 + (-1) \cdot \sin(x) \cdot 0) \cdot 0\]
Согласно школьной программе:
\[ \frac{d}{dx}(((0 \cdot \sin(x) + (-1) \cdot \cos(x) \cdot 1) \cdot 1 + (-1) \cdot \sin(x) \cdot 0) \cdot 1 + (-1) \cdot \sin(x) \cdot 1 \cdot 0) = ((0 \cdot \sin(x) + 0 \cdot \cos(x) \cdot 1 + 0 \cdot \cos(x) \cdot 1 + (-1) \cdot ((-1) \cdot \sin(x) \cdot 1 \cdot 1 + \cos(x) \cdot 0)) \cdot 1 + (0 \cdot \sin(x) + (-1) \cdot \cos(x) \cdot 1) \cdot 0 + (0 \cdot \sin(x) + (-1) \cdot \cos(x) \cdot 1) \cdot 0 + (-1) \cdot \sin(x) \cdot 0) \cdot 1 + ((0 \cdot \sin(x) + (-1) \cdot \cos(x) \cdot 1) \cdot 1 + (-1) \cdot \sin(x) \cdot 0) \cdot 0 + ((0 \cdot \sin(x) + (-1) \cdot \cos(x) \cdot 1) \cdot 1 + (-1) \cdot \sin(x) \cdot 0) \cdot 0 + (-1) \cdot \sin(x) \cdot 1 \cdot 0\]
Произведя некоторые подстановки:
\[ \frac{d}{dx}(((0 \cdot \sin(x) + (-1) \cdot \cos(x) \cdot 1) \cdot 1 + (-1) \cdot \sin(x) \cdot 0) \cdot 1 + (-1) \cdot \sin(x) \cdot 1 \cdot 0 + (-1) \cdot \sin(x) \cdot 1 \cdot 0 + \cos(x) \cdot 0) = ((0 \cdot \sin(x) + 0 \cdot \cos(x) \cdot 1 + 0 \cdot \cos(x) \cdot 1 + (-1) \cdot ((-1) \cdot \sin(x) \cdot 1 \cdot 1 + \cos(x) \cdot 0)) \cdot 1 + (0 \cdot \sin(x) + (-1) \cdot \cos(x) \cdot 1) \cdot 0 + (0 \cdot \sin(x) + (-1) \cdot \cos(x) \cdot 1) \cdot 0 + (-1) \cdot \sin(x) \cdot 0) \cdot 1 + ((0 \cdot \sin(x) + (-1) \cdot \cos(x) \cdot 1) \cdot 1 + (-1) \cdot \sin(x) \cdot 0) \cdot 0 + ((0 \cdot \sin(x) + (-1) \cdot \cos(x) \cdot 1) \cdot 1 + (-1) \cdot \sin(x) \cdot 0) \cdot 0 + (-1) \cdot \sin(x) \cdot 1 \cdot 0 + ((0 \cdot \sin(x) + (-1) \cdot \cos(x) \cdot 1) \cdot 1 + (-1) \cdot \sin(x) \cdot 0) \cdot 0 + (-1) \cdot \sin(x) \cdot 1 \cdot 0 + (-1) \cdot \sin(x) \cdot 1 \cdot 0 + \cos(x) \cdot 0\]
{\Large В итоге:}

\[\boxed{ \sin(x) = 0 + \frac{0}{1} \cdot x^{0} + \frac{1}{1} \cdot x^{1} + \frac{0}{2} \cdot x^{2} + \frac{(-1)}{6} \cdot x^{3}+ o(x^3)}\]
{\Large Теперь упростим полученнoе выражение:}

Наведем косметики в функции:
\[ f = 0 + \frac{0}{1} \cdot x^{0} + \frac{1}{1} \cdot x^{1} + \frac{0}{2} \cdot x^{2} + \frac{(-1)}{6} \cdot x^{3} \]

Все доказано:
\[ \frac{0}{1} = 0\]
Ввиду нехитрых преобразований:
\[ \frac{1}{1} = 1\]
Согласно школьной программе:
\[ \frac{0}{2} = 0\]
Нетрудно заметить, что:
\[ \frac{(-1)}{6} = (-0.166667)\]
Для любых из 6 СПС, верно, что:
(Желаю всем, кто пишет "СПС" вместо "спасибо"
продуктивно потратить сэкономленное время)
\[0 \cdot x^{0} = 0 \]

Произведя некоторые подстановки:
\[0 + 0 = 0 \]

Внимательный читатель заметит, что
\[1 \cdot x^{1} = x^{1} \]

Очевидно что:
\[0 + x^{1} = x^{1} \]

Очевидно что:
\[0 \cdot x^{2} = 0 \]

Все доказано:
\[x^{1} + 0 = x^{1} \]

Итого:
\[ x^{1} + (-0.166667) \cdot x^{3} \]

{\Large Вcё доказано:}

\[\boxed{ \sin(x) = x^{1} + (-0.166667) \cdot x^{3}+ o(x^3)}\]
\end{document}
