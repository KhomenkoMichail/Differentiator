\documentclass{article}
\usepackage[T2A]{fontenc}
\usepackage[utf8]{inputenc}
\usepackage[russian,english]{babel}
\usepackage{amsmath}
\usepackage{amssymb}
\usepackage{graphicx}
\usepackage{geometry}
\geometry{a4paper, margin=0.5cm}
\begin{document}

\begin{center}
\vspace*{3cm}

{\Huge \textbf{Курс \guillemotleftОт раздолб до отл 10 на семестрой по матану за одну ночь\guillemotright.}}\\[1cm]
{\Large Автор: Хоменко М.М.}\\[0.5cm]
\vfill
{\Large Долгопрудный, \the\year}
\end{center}
\newpage

\section{Введение}
Семестровая по матану лишь одно из жизненных испытаний, которое вам предстоит пройти. Будьте уверены: каждому, кто учился в школе, по силам сдать семестровую. Все задания составлены на основе школьной программы. Поэтому каждый из вас может успешно сдать семестровую.

Автор данного пособия не учился в школе и, дабы помочь себе подобным сдать экзамен, собрал небольшой курс упражнений для подготовки всего за одну ночь. Желаю Вам приятного времяпрепровождения.

\section{Упражнение первое: взятие производной простейшей функции}
Имеем функцию:
\[\boxed{ f = \sin((\frac{1}{x})) \cdot \cos((2 \cdot x + 1))^{3} - \frac{\ln((x^{2}))}{e^{x}} + \tan((3.14 \cdot x)) }\]

 Ее график имеет вид:

\begin{figure}[h]
	\centering
	\includegraphics[width=0.5\textwidth]{FUNC_GRAPHS/funcGraph1.png}
	\caption{график исходной функции}
\end{figure}
{\Large Вычислим производную данной функции:}

Из леммы 6.66, следует, что:
\[ \frac{d}{dx}(x) = 1\]
Из леммы 6.66, следует, что:
\[ \frac{d}{dx}(3.14) = 0\]
Для любых из 6 СПС, верно, что:
(Желаю всем, кто пишет "СПС" вместо "спасибо"
продуктивно потратить сэкономленное время)
\[ \frac{d}{dx}((3.14 \cdot x)) = 0 \cdot x + 3.14 \cdot 1\]
Нетрудно заметить, что:
\[ \frac{d}{dx}(\tan((3.14 \cdot x))) = \frac{1}{\cos((3.14 \cdot x))^{2}} \cdot (0 \cdot x + 3.14 \cdot 1)\]
Согласно школьной программе:
\[ \frac{d}{dx}(x) = 1\]
Внимательный читатель заметит, что
\[ \frac{d}{dx}(e^{x}) = e^{x} \cdot 1\]
Все доказано:
\[ \frac{d}{dx}(x) = 1\]
Нетрудно заметить, что:
\[ \frac{d}{dx}((x^{2})) = 2 \cdot x^{(2 - 1)} \cdot 1\]
Согласно школьной программе:
\[ \frac{d}{dx}(\ln((x^{2}))) = \frac{1}{x^{2}} \cdot 2 \cdot x^{(2 - 1)} \cdot 1\]
Произведя некоторые подстановки:
\[ \frac{d}{dx}(\frac{\ln((x^{2}))}{e^{x}}) = \frac{(\frac{1}{x^{2}} \cdot 2 \cdot x^{(2 - 1)} \cdot 1 \cdot e^{x} - \ln((x^{2})) \cdot e^{x} \cdot 1)}{e^{x} \cdot e^{x}}\]
Для любых из 6 СПС, верно, что:
(Желаю всем, кто пишет "СПС" вместо "спасибо"
продуктивно потратить сэкономленное время)
\[ \frac{d}{dx}(1) = 0\]
Для любых из 6 СПС, верно, что:
(Желаю всем, кто пишет "СПС" вместо "спасибо"
продуктивно потратить сэкономленное время)
\[ \frac{d}{dx}(x) = 1\]
Ввиду нехитрых преобразований:
\[ \frac{d}{dx}(2) = 0\]
Все доказано:
\[ \frac{d}{dx}(2 \cdot x) = 0 \cdot x + 2 \cdot 1\]
Произведя некоторые подстановки:
\[ \frac{d}{dx}((2 \cdot x + 1)) = 0 \cdot x + 2 \cdot 1 + 0\]
Не умаляя общности:
\[ \frac{d}{dx}(\cos((2 \cdot x + 1))) = (-1) \cdot \sin((2 \cdot x + 1)) \cdot (0 \cdot x + 2 \cdot 1 + 0)\]
Внимательный читатель заметит, что
\[ \frac{d}{dx}(\cos((2 \cdot x + 1))^{3}) = 3 \cdot \cos((2 \cdot x + 1))^{(3 - 1)} \cdot (-1) \cdot \sin((2 \cdot x + 1)) \cdot (0 \cdot x + 2 \cdot 1 + 0)\]
Ввиду нехитрых преобразований:
\[ \frac{d}{dx}(x) = 1\]
Согласно школьной программе:
\[ \frac{d}{dx}(1) = 0\]
Нетрудно заметить, что:
\[ \frac{d}{dx}((\frac{1}{x})) = \frac{(0 \cdot x - 1 \cdot 1)}{x \cdot x}\]
Нетрудно заметить, что:
\[ \frac{d}{dx}(\sin((\frac{1}{x}))) = \cos((\frac{1}{x})) \cdot \frac{(0 \cdot x - 1 \cdot 1)}{x \cdot x}\]
Ввиду нехитрых преобразований:
\[ \frac{d}{dx}(\sin((\frac{1}{x})) \cdot \cos((2 \cdot x + 1))^{3}) = \cos((\frac{1}{x})) \cdot \frac{(0 \cdot x - 1 \cdot 1)}{x \cdot x} \cdot \cos((2 \cdot x + 1))^{3} + \sin((\frac{1}{x})) \cdot 3 \cdot \cos((2 \cdot x + 1))^{(3 - 1)} \cdot (-1) \cdot \sin((2 \cdot x + 1)) \cdot (0 \cdot x + 2 \cdot 1 + 0)\]
Ввиду нехитрых преобразований:
\[ \frac{d}{dx}(\sin((\frac{1}{x})) \cdot \cos((2 \cdot x + 1))^{3} - \frac{\ln((x^{2}))}{e^{x}}) = \cos((\frac{1}{x})) \cdot \frac{(0 \cdot x - 1 \cdot 1)}{x \cdot x} \cdot \cos((2 \cdot x + 1))^{3} + \sin((\frac{1}{x})) \cdot 3 \cdot \cos((2 \cdot x + 1))^{(3 - 1)} \cdot (-1) \cdot \sin((2 \cdot x + 1)) \cdot (0 \cdot x + 2 \cdot 1 + 0) - \frac{(\frac{1}{x^{2}} \cdot 2 \cdot x^{(2 - 1)} \cdot 1 \cdot e^{x} - \ln((x^{2})) \cdot e^{x} \cdot 1)}{e^{x} \cdot e^{x}}\]
Из леммы 6.66, следует, что:
\[ \frac{d}{dx}(\sin((\frac{1}{x})) \cdot \cos((2 \cdot x + 1))^{3} - \frac{\ln((x^{2}))}{e^{x}} + \tan((3.14 \cdot x))) = \cos((\frac{1}{x})) \cdot \frac{(0 \cdot x - 1 \cdot 1)}{x \cdot x} \cdot \cos((2 \cdot x + 1))^{3} + \sin((\frac{1}{x})) \cdot 3 \cdot \cos((2 \cdot x + 1))^{(3 - 1)} \cdot (-1) \cdot \sin((2 \cdot x + 1)) \cdot (0 \cdot x + 2 \cdot 1 + 0) - \frac{(\frac{1}{x^{2}} \cdot 2 \cdot x^{(2 - 1)} \cdot 1 \cdot e^{x} - \ln((x^{2})) \cdot e^{x} \cdot 1)}{e^{x} \cdot e^{x}} + \frac{1}{\cos((3.14 \cdot x))^{2}} \cdot (0 \cdot x + 3.14 \cdot 1)\]
{\Large Теперь упростим полученную производную:}

Наведем косметики в функции:
\[ f = \cos((\frac{1}{x})) \cdot \frac{(0 \cdot x - 1 \cdot 1)}{x \cdot x} \cdot \cos((2 \cdot x + 1))^{3} + \sin((\frac{1}{x})) \cdot 3 \cdot \cos((2 \cdot x + 1))^{(3 - 1)} \cdot (-1) \cdot \sin((2 \cdot x + 1)) \cdot (0 \cdot x + 2 \cdot 1 + 0) - \frac{(\frac{1}{x^{2}} \cdot 2 \cdot x^{(2 - 1)} \cdot 1 \cdot e^{x} - \ln((x^{2})) \cdot e^{x} \cdot 1)}{e^{x} \cdot e^{x}} + \frac{1}{\cos((3.14 \cdot x))^{2}} \cdot (0 \cdot x + 3.14 \cdot 1) \]

Согласно школьной программе:
\[ 1 \cdot 1 = 1\]
Для любых из 6 СПС, верно, что:
(Желаю всем, кто пишет "СПС" вместо "спасибо"
продуктивно потратить сэкономленное время)
\[ (3 - 1) = 2\]
Согласно школьной программе:
\[ 2 \cdot 1 = 2\]
Нетрудно заметить, что:
\[ (2 - 1) = 1\]
Произведя некоторые подстановки:
\[ 3.14 \cdot 1 = 3.14\]
Согласно школьной программе:
\[0 \cdot x = 0 \]

Очевидно что:
\[0 \cdot x = 0 \]

Произведя некоторые подстановки:
\[0 + 2 = 2 \]

Произведя некоторые подстановки:
\[(2 + 0) = 2 \]

Согласно школьной программе:
\[2 \cdot x^{1} \cdot 1 = 2 \cdot x^{1} \]

Все доказано:
\[e^{x} \cdot 1 = e^{x} \]

Очевидно что:
\[0 \cdot x = 0 \]

Нетрудно заметить, что:
\[(0 + 3.14) = 3.14 \]

Согласно школьной программе:
\[ (0 - 1) = (-1)\]
Итого:
\[ \cos((\frac{1}{x})) \cdot \frac{(-1)}{x \cdot x} \cdot \cos((2 \cdot x + 1))^{3} + \sin((\frac{1}{x})) \cdot 3 \cdot \cos((2 \cdot x + 1))^{2} \cdot (-1) \cdot \sin((2 \cdot x + 1)) \cdot 2 - \frac{(\frac{1}{x^{2}} \cdot 2 \cdot x^{1} \cdot e^{x} - \ln((x^{2})) \cdot e^{x})}{e^{x} \cdot e^{x}} + \frac{1}{\cos((3.14 \cdot x))^{2}} \cdot 3.14 \]

{\Large Итого получаем:}

\[\boxed{ \frac{d}{dx}(\sin((\frac{1}{x})) \cdot \cos((2 \cdot x + 1))^{3} - \frac{\ln((x^{2}))}{e^{x}} + \tan((3.14 \cdot x))) = \cos((\frac{1}{x})) \cdot \frac{(-1)}{x \cdot x} \cdot \cos((2 \cdot x + 1))^{3} + \sin((\frac{1}{x})) \cdot 3 \cdot \cos((2 \cdot x + 1))^{2} \cdot (-1) \cdot \sin((2 \cdot x + 1)) \cdot 2 - \frac{(\frac{1}{x^{2}} \cdot 2 \cdot x^{1} \cdot e^{x} - \ln((x^{2})) \cdot e^{x})}{e^{x} \cdot e^{x}} + \frac{1}{\cos((3.14 \cdot x))^{2}} \cdot 3.14}\]
{\Large График полученной производной:}

\begin{figure}[h]
	\centering
	\includegraphics[width=0.5\textwidth]{FUNC_GRAPHS/funcGraph2.png}
	\caption{график производной}
\end{figure}
\section{Упражнение второе: вычисление касательной функции в точке}
Уравнение касательной функции:
\[ f = \sin((\frac{1}{x})) \cdot \cos((2 \cdot x + 1))^{3} - \frac{\ln((x^{2}))}{e^{x}} + \tan((3.14 \cdot x)) \]

 в точке x = 3:
\[\boxed{ f = 0.0260297 + (2.43822)*(x - 3)}\]

\begin{figure}[h]
	\centering
	\includegraphics[width=0.5\textwidth]{FUNC_GRAPHS/funcGraph3.png}
	\caption{График касательной функции в точке}
\end{figure}
\end{document}
