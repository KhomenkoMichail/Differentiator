\documentclass{article}
\usepackage[T2A]{fontenc}
\usepackage[utf8]{inputenc}
\usepackage[russian,english]{babel}
\usepackage{amsmath}
\usepackage{amssymb}
\usepackage{graphicx}
\usepackage{geometry}
\geometry{a4paper, margin=0.5cm}
\begin{document}

\begin{center}
\vspace*{3cm}

{\Huge \textbf{Курс С раздолб до отл 10 на семестрой по матану за одну ночь.}}\\[1cm]
{\large Выполнил: Хоменко М.М.}\\[0.5cm]
\vfill
{\large Долгопрудный, \the\year}
\end{center}
\newpage

\section{Введение}
Семестровая по матану лишь одно из жизненных испытаний, которое вам предстоит пройти. Будьте уверены: каждому, кто учился в школе, по силам сдать семестровую. Все задания составлены на основе школьной программы. Поэтому каждый из вас может успешно сдать семестровую.

Автор данного пособия не учился в школе и, дабы помочь себе подобным сдать экзамен, собрал небольшой курс упражнений для подготовки всего за одну ночь. Желаю вам приятного времяпрепровождения.

\section{Задание первое: взятие производной простейшей функции}
Имеем функцию:
\[ f = \sin(x) \]

Ввиду нехитрых преобразований:
\[ \frac{d}{dx}(x) = 1\]
Внимательный читатель заметит, что
\[ \frac{d}{dx}(\sin(x)) = \cos(x) \cdot 1\]
Наведем косметики в функции:
\[ f = \cos(x) \cdot 1 \]

Для любых из 6 СПС, верно, что:
(Желаю всем, кто пишет "СПС" вместо "спасибо"
продуктивно потратить сэкономленное время)
\[\cos(x) \cdot 1 = \cos(x) \]

Итого:
\[ \cos(x) \]

\begin{figure}[h]
	\centering
	\includegraphics[width=0.5\textwidth]{FUNC_GRAPHS/funcGraph1.png}
	\caption{график исходной функции}
\end{figure}
\begin{figure}[h]
	\centering
	\includegraphics[width=0.5\textwidth]{FUNC_GRAPHS/funcGraph2.png}
	\caption{график производной}
\end{figure}
\section{Нахождение касательной функции в точке.}
Уравнение касательной функции:
\[ f = \sin(x) \]

 в точке x = 1:
\[ f = 0.841471 + (0.540302)*(x - 1)\]

\begin{figure}[h]
	\centering
	\includegraphics[width=0.5\textwidth]{FUNC_GRAPHS/funcGraph3.png}
\end{figure}
\end{document}
