\documentclass{article}
\usepackage[utf8]{inputenc}
\usepackage[russian]{babel}
\usepackage{amsmath}
\usepackage{amssymb}
\usepackage{graphicx}
\usepackage{geometry}
\geometry{a4paper, margin=0.5cm}
\begin{document}
Все что мы имеем:
\[ f = \sin(x) \]

Внимательный читатель заметит, что
\[ \frac{d}{dx}(x) = 1\]
Произведя некоторые подстановки:
\[ \frac{d}{dx}(\sin(x)) = \cos(x) \cdot 1\]
Наведем косметики в функции:
\[ f = \cos(x) \cdot 1 \]

Из леммы 6.66, следует, что:
\[\cos(x) \cdot 1 = \cos(x) \]

Итого:
\[ \cos(x) \]

\begin{figure}[h]
	\centering
	\includegraphics[width=0.5\textwidth]{FUNC_GRAPHS/funcGraph1.png}
	\caption{график исходной функции}
\end{figure}
\begin{figure}[h]
	\centering
	\includegraphics[width=0.5\textwidth]{FUNC_GRAPHS/funcGraph2.png}
	\caption{график производной}
\end{figure}
\section{Нахождение касательной функции в точке.}
Уравнение касательной функции:
\[ f = \sin(x) \]

 в точке x = 2:
\[ f = 0.909297 + (-0.416147)*(x - 2)\]

\begin{figure}[h]
	\centering
	\includegraphics[width=0.5\textwidth]{FUNC_GRAPHS/funcGraph3.png}
\end{figure}
\end{document}
